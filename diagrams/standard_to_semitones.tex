\documentclass[pstricks]{amsart}
\usepackage{amsmath}
\usepackage{amssymb}
\usepackage{multirow}
\usepackage{hyperref}
\usepackage{tikz,pgf}
\usepackage[margin=.5in]{geometry}
\usetikzlibrary{calc}

\usetikzlibrary{intersections, calc, fpu, decorations.pathreplacing}

\def\major{2,2,1,2,2,2,1}
\def\harmonic{2, 1, 2, 2, 1, 3, 1}
\def\melodic{2, 1, 2, 2, 2, 2, 1}

\author{cuppajoeman}
\thanks{test}

\begin{document}

Conversion from standard notation to semitones - Made by cuppajoeman - \url{http://cuppajoeman.com}

\[ 
  \huge
\begin{array}{cccccccccccc}
    C& \cdot& D& \cdot& E& F& \cdot& G& \cdot& A& \cdot& B \\
    \updownarrow & \updownarrow & \updownarrow & \updownarrow & \updownarrow & \updownarrow & \updownarrow & \updownarrow & \updownarrow & \updownarrow & \updownarrow & \updownarrow \\
    0 & 1 & 2 & 3 & 4 & 5 & 6 & 7 & 8 & 9 & 10 & 11 \\
\end{array} 
\] 

\vspace{.15cm}

\begin{center}
\begin{tikzpicture}
  % Ledger lines
  \draw (13.25,6) -- (17,6);
  % Main section
  \draw (8.5,5) -- (17,5);
  \foreach \y in {0, 1, 2, 3, 4} {
    \draw (0, \y) -- (17, \y);
  }
  \draw (0,0) -- (0, 4);
  \draw (17,0) -- (17, 4);
  %\draw [step=1.0,black] (0,0) grid (17,4);
  % Lower lines
  \draw (0,-1) -- (8.5,-1);
  \draw (0,-2) -- (4.75,-2);
  % Counter
  \edef\counter{0}
  \foreach \x in {10, 11, 0, 2, 4, 5, 7, 9, 11, 0, 2, 4, 5, 7 , 9, 11, 0} {
    %\node[draw,circle,fill=white,minimum size=2] at (\counter + .5, \counter * .5 - 2) {\huge\x};
    \node[text=cyan] at (\counter + .5, \counter * .5 - 2) {\Huge\x};
    \pgfmathparse{\counter+1}
    \xdef\counter{\pgfmathresult}
  }
    %\draw at (\x + .5, \x * .5) {\x}
\end{tikzpicture}
\end{center}

\vspace{.15cm}

\begin{center}
\begin{tikzpicture}
  \foreach \a [count=\c,evaluate=\c as \y using {{\major}[\c-1]}] in  {1,2,...,7}{
    \draw (\a*360/7*-1: 2.5) node {\huge{\y}};
  }
  \node[draw] at (0,0) {Major/Minor $\circlearrowright$};
\end{tikzpicture}
\hfill
\begin{tikzpicture}
  \foreach \a [count=\c,evaluate=\c as \y using {{\harmonic}[\c-1]}] in  {1,2,...,7}{
    \draw (\a*360/7*-1: 2.5) node {\huge{\y}};
  }
  \node[draw] at (0,0) {Harmonic $\circlearrowright$};
\end{tikzpicture}\hfill
\begin{tikzpicture}
  \foreach \a [count=\c,evaluate=\c as \y using {{\melodic}[\c-1]}] in  {1,2,...,7}{
    \draw (\a*360/7*-1: 2.5) node {\huge{\y}};
  }
  \node[draw] at (0,0) {Melodic $\circlearrowright$};
\end{tikzpicture}
\end{center}

\vspace{.15cm}

% Please add the following required packages to your document preamble:
% \usepackage{multirow}
\begin{table}[h]
  \centering
  \begin{tabular}{|c|c|c|c|}
    \hline
    \begin{tabular}[c]{@{}c@{}}Number of\\ semitones\end{tabular} & Name (Minor, Major or Perfect) & Short             & Name (Diminished or Augmented) \\ \hline
    0                                                             & Perfect unison                 & P1                & Diminished second              \\ \hline
    1                                                             & Minor second                   & m2                & Augmented unison               \\ \hline
    2                                                             & Major second                   & M2                & Diminished third               \\ \hline
    3                                                             & Minor third                    & m3                & Augmented second               \\ \hline
    4                                                             & Major third                    & M3                & Diminished fourth              \\ \hline
    5                                                             & Perfect fourth                 & P4                & Augmented third                \\ \hline
    \multirow{2}{*}{6}                                            & \multirow{2}{*}{}              & \multirow{2}{*}{} & Tritone|Diminished fifth       \\ \cline{4-4} 
                                                                  &                                &                   & Tritone|Augmented fourth       \\ \hline
    7                                                             & Perfect fifth                  & P5                & Diminished sixth               \\ \hline
    8                                                             & Minor sixth                    & m6                & Augmented fifth                \\ \hline
    9                                                             & Major sixth                    & M6                & Diminished seventh             \\ \hline
    10                                                            & Minor seventh                  & m7                & Augmented sixth                \\ \hline
    11                                                            & Major seventh                  & M7                & Diminished octave              \\ \hline
    12                                                            & Octave|Perfect octave          & P8                & Augmented seventh              \\ \hline
  \end{tabular}
\end{table}






\end{document}
